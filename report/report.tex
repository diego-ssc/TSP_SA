\documentclass[a4paper]{report}

\usepackage{pgf}
\usepackage[utf8]{inputenc}
\usepackage{verbatim}
\usepackage{titling}
\usepackage{booktabs}
\usepackage{enumitem}
\usepackage{qtree}
\usepackage{amssymb}
\usepackage{amsmath}
\usepackage{times}
\usepackage{dsfont}
\usepackage{titling}
\usepackage[spanish]{babel}
% \usepackage{svg}
% \svgpath{../data/plot}


\pretitle{\begin{center}\linespread{1}}
  \posttitle{\end{center}\vspace{0.14cm}}
\preauthor{\begin{center}\Large}
  \postauthor{\end{center}}

\setlength{\droptitle}{-10em}
\title {\textbf {\Large{El problema del agente viajero}}\protect\\
  \large{\textbf{Recocido Simulado}}\protect\\ \vspace{0.4cm}
  \normalsize{\textbf{Seminario de Ciencias de la Computaci\'on B}} \protect\\ \vspace{0.2cm}
  \normalsize{Canek Pel\'aez Vald\'es}}
\date{}
\author{\normalsize Sánchez Correa Diego Sebastián}

\begin{document}
\allowdisplaybreaks
\maketitle
\tableofcontents

\chapter{Introducci\'on}
\section{Heur\'istica}
\chapter{Dise\~no}

El dise\~no sigui\'o una estructura orientada a objetos. Contar con la informaci\'on
de las ciudades en una base de datos relacional, sugiri\'o la implementaci\on de
un objeto \texttt{database_loader}. De la msima manera, al contar con soluciones
compuestas por permutaciones de ciudades, fue claro el uso de las clases \texttt{city}
y \texttt{path} (ciudad y camino, respectivamente).

\chapter{Implementaci\'on}
\chapter{Experimentaci\'on}

\begin{table}[h!]
  \begin{center}
    \begin{tabular}{||l|c||}
      \hline
      \textit{Par\'ametro} & \textit{Valor}\\
      \hline
      Semilla & 102 \\
      Temperatura & 14 \\
      M & 122000 \\
      L & 1200 \\
      \'Epsilon & 0.002 \\
      Phi & 0.95 \\
      \hline
    \end{tabular}
  \end{center}
\end{table}


% \begin{figure}[h!tbp]
%   \hspace*{-1.6cm}
%   \includesvg[scale=1]{results_1}
% \end{figure}


% \begin{figure}[h!tbp]
%   \hspace*{-1.6cm}
%   \includesvg[scale=1]{results_2}
% \end{figure}

\chapter{Conclusiones}

\section{Bibliograf\'ia}


\end{document}
