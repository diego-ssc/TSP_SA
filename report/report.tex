\documentclass[a4paper]{report}

\usepackage{pgf}
\usepackage[utf8]{inputenc}
\usepackage{verbatim}
\usepackage{titling}
\usepackage{booktabs}
\usepackage{enumitem}
\usepackage{qtree}
\usepackage{amssymb}
\usepackage{amsmath}
\usepackage{times}
\usepackage{dsfont}
\usepackage{titling}
\usepackage{cite}
\usepackage[spanish]{babel}
\usepackage{svg}
\svgpath{../data/plot}


\pretitle{\begin{center}\linespread{1}}
  \posttitle{\end{center}\vspace{0.14cm}}
\preauthor{\begin{center}\Large}
  \postauthor{\end{center}}

\setlength{\droptitle}{-10em}
\title {\textbf {\Large{El problema del agente viajero}}\protect\\
  \large{\textbf{Recocido Simulado}}\protect\\ \vspace{0.4cm}
  \normalsize{\textbf{Seminario de Ciencias de la Computaci\'on B}} \protect\\ \vspace{0.2cm}
  \normalsize{Canek Pel\'aez Vald\'es} \protect\\ \vspace{0.4cm}
  \normalsize{Universidad Nacional Aut\'onoma de M\'exico}}
\date{}
\author{\normalsize Sánchez Correa Diego Sebastián}


\begin{document}
\allowdisplaybreaks
\maketitle
\tableofcontents

\chapter{Introducci\'on}
Podr\'iamos pensar que el mundo actual, dada la tecnolog\'ia con la que contamos,
cuenta con computadoras capaces de hasta la m\'as grande de las computaciones. Vaya que es
cierto, que, desde el surgimento de la computadora, el progreso tecnol\'ogico ha sido inmenso,
sin embargo, sin importar el poder computacional al que hemos llegado, siguien existiendo
problemas que no son resolubles en un tiempo racionalmente corto, unos problemas que, incluso,
podr\'ian definirse como irresolubles por el tiempo que conllevar\'ia computarlos para obtener
la mejor soluci\'on.\\

Los tiempos de computaci\'on de cualquier algoritmo caer\'an en dos grupos.
El primero, consiste de solo los problemas cuyas soluciones est\'an acotadas por un polinomio
de un grado bajo. El segundo, por los problemas cuyos mejores algoritmos son no polinomiales.

Estos \'ultimos, son los problemas NP. Los problemas \textit{NP} (Non Deterministic Polynimial Time)
pueden ser subdivididos en \textit{NP-Completo} y \textit{NP-Duro}.

Un problema que es \textit{NP-Completo} tiene la propiedad de que podr\'a ser resulto en tiempo
polinomial si y solo si todos los problemas en el conjunto \textit{NP-Completo} pueden ser
resueltos en tiempo polinomial. Adem\'as, existe un algoritmo que verifica que una soluci\'on
dada es la \'optima. Esta \'ultima propiedad hace destacar a los problemas \textit{NP-duro} y este
el conjunto donde se encuentra el Problema del Agente Viajero.
\clearpage
\section{El problema}

\begin{center}
  \textit{...la pregunta es encontrar, para un conjunto finito de puntos de los
    cuales se conocen las distancias entre cada par, el camino más corto entre
    estos puntos. Por supuesto, el problema es resuelto por un conjunto finito
    de intentos. \textbf{Schrijver (2005)}
    \footnote{Alexander (Lex) Schrijver (4 de Mayo 1948, \'Amsterdam) es
      un matem\'atico y cient\'ifico en computaci\'on holand\'es,
      profesor de matem\'aticas discretas y optimizaci\'on
      en la Universidad de \'Amsterdam}}
\end{center}

\begin{center}
  \textit{Dados n puntos y todas las distancias entre las parejas ordenadas de
    distintos puntos. El problema es hallar un sistema de circuitos ordenados tal que:\\
    1. Cada punto permanezca en exactamente un circuito.\\
    2. Cada circuito contenga al menos 2 puntos.\\
    3. Ning\'un circuito pase por a trav\'es del mismo punto m\'as de una vez.\\
    4. La distancia total de los circuitos es un m\'inimo.
  }~\cite{julia}
\end{center}


\section{La Heur\'istica}
El recocido simulado \textit{es un m\'etodo probabil\'istico propuesto por Kirkpatrick, Gelett,
  y Vecchi y Cerny para encontar el m\'inimo global de una funci\'on de costo que puede poseer
  varios m\'inimos locales~\cite{dimitris}.}

Esta fue formulada a partir de la simulaci\'on de un fen\'omeno f\'isico usado com\'unmente
en la metalurgia, se trata del proceso que se da, una vez calentado un metal y enfriado, posteriormente,
si este no es vuelto a calentar, entonces, este se har\'a muy duro y fr\'agil, en cambio, el recocido
pretende dar maleabilidad y ductilidad al material enfriando en peque\~nos umbrales el material.

\chapter{Dise\~no}

El dise\~no sigui\'o una estructura orientada a objetos. Contar con la informaci\'on
de las ciudades en una base de datos relacional, sugiri\'o la implementaci\'on de
un objeto \textbf{\texttt{database\_loader}}. De la msima manera, el contar con soluciones
compuestas por permutaciones de ciudades, hizo claro el uso de las clases
\textbf{\texttt{city}} y \textbf{\texttt{path}} (ciudad y camino, respectivamente).
De la misma manera, se ha planteado el modelado del problema y de la heur\'istica
como clases compuestas a partir las unidades m\'as fundamentales y con atributos y
comportamiento asociado.\\

\section{La base de datos}
El problema se plante\'o como una base de datos relacional cuyo contenido est\'a
compuesto de las ciudades (1092 en total) y de sus conexiones (que representan
la gr\'afica del problema).\\

Es importante mencionar que la matriz de adyacencias represnetada por la tabla de
conexiones no es cuadrada, sin embargo, el objeto \textbf{\texttt{database\_loader}}
la cargar\'a  a una matriz de enteros (\textbf{\texttt{int}} de C) haci\'endola cuadrada.\\

La tabla de ciudades se carga a un arreglo de apuntadores a objetos de tipo
\textbf{\texttt{city}}. Normalmente se usar\'ia un puntero a un objeto de tipo
\textbf{\texttt{city}}, pero el constructor de la clase, naturalmente, aloja los
objetos en el heap, devolviendo, por lo tanto, un puntero a un objeto;
por ello, el arreglo solo re\'une los punteros de los objetos que
ya se encuentran guardados din\'amicamente; evitando, en conjunto,
una posible ambig\"uedad al tratar con ambos tipos que, en esencia, no
comparten caracter\'isticas adicionales a ser un apuntador a una ciudad.

\section{Las ciudades}
El objeto \textbf{\texttt{city}} tiene como propiedades los campos incluidos en la base de
datos, con excepci\'on de la poblaci\'on. La clase abstrae el concepto de una
ciudad, omitiendo una representaci\'on total a partir de atributos asociados
(como lo hace la base de datos), sino involucrando en la composici\'on un
comportamiento particular.\\

Esta abstracci\'on se da al decidir que una ciudad proporcione el comportamiento
para calcular la distancia con otra ciudad en conjunto con su uso como unidad
fundamental del problema.\\

Cabe destacar que la clase \textbf{\texttt{city}} tambi\'en es la encargada de crear los
arreglos de ciudades, resultado de una "modularizaci\'on" del dise\~no : ninguna clase
conoce el tama\~no de la estructura \textbf{\texttt{city}}, por lo tanto, ninguna clase (a
excepci\'on de city) es capaz de crear din\'amicamente un arreglo de ciudades; esto
podr\'ia evitarse si se incluyera la declaraci\'on de la estructura dentro del encabezado
(header) de la clase, pero considero se tratar\'ia de una violaci\'on a la encapsulaci\'on
de los datos.

\section{Los caminos}
Al tener la idea de una ciudad ya abstra\'ida en una clase, era natural que los caminos consitieran
de, al menos, un arreglo de ciudades. El primer dise\~no que realic\'e se trataba de una representaci\'on
sin atributos, ni comportamiento; solo un arreglo de ciudades. Poco tiempo despu\'es escrib\'i la
clase \textbf{\texttt{path}}. Esta ahora no solo se trataba de un arreglo de ciudades, sino de
una estructura con atributos y comportamiendo asociados: esto facilit\'o (desde un punto de vista
del disen\~no) la implementaciones de funciones como el intercambio de ciudades dentro de un camino,
el c\'alculo del normalizador y la funci\'on de costo.\\

La creaci\'on de un camino implica el c\'alculo de la distancia m\'axima de esa instancia, as\'i, como el
normalizador y la suma de los costos de sus ciudades. Estos atributos se calculan a partir de operaciones
lineales, por ello, son valores que incializo dentro del constructor de la clase y, si se quiere acceder
a ellos una vez creado al objeto, la operaci\'on se reducir\'a a una consulta lineal del atributo deseado
del objeto. Esta soluci\'on implicaba la disminuci\'on de la cantidad de invocaciones de funciones para
crear una camino v\'alido (dej\'o de ser necesario invocar la funci\'on, seguido de usar un setter del
objeto para guardar el resultado), sin embargo, me imped\'ia utilizar el mismo constructor para hacer copias
de un objeto (usadas en la heur\'istica) si pretend\'ia mantener el tiempo de ejecuci\'on en un estado
\'optimo, por supuesto.\\

La implementaci\'on de la funci\'on copia de la clase \textbf{\texttt{path}}, entonces, no invoca al
constructor de la clase, sino que accede a las propiedades del objeto par\'ametro y aquellas que eran
computadas linealmente, se asociar\'ian como copias al nuevo objeto.

\section{TSP}
El problema del agente viajero fue, como lo han sido todas las estructuras del proyecto, modelado siguiendo
un dise\~no orientado a objetos, sin embargo, no ser\'ia un error afirmar que este no era su prop\'osito inicial.
La clase comenz\'o siendo un objeto central de ejecuci\'on que administrar\'ia la invocaci\'on y liberaci\'on
de todas las clases necesarias, incluso, ser\'ia la encargada de ejecutar la heur\'istica.\\

La clase desempe\~na el papel de marcar el inicio de la ejecuci\'on del problema, creando un objeto
\textbf{\texttt{database\_loader}}, proveyendo la informaci\'on computada en este.
Tiene como atributo un camino (el \'unico creado en la ejecuci\'on del programa; con el constructor
de la clase, al menos) que tambi\'en ser\'a usado para la heur\'istica. Por ello, una ejeuci\'on,
no estar\'a completa sin una invocaci\'on a la clase \textbf{\texttt{sa}}.

\section{Recocido Simulado}
La clase implementa el algoritmo del Recocido Simulado guarda los par\'ametros que usa la heur\'istica
(adem\'as de definir unos predeterminados) y est\'a asociada con \textbf{\texttt{tsp}} teni\'endola
como un atributo; el camino inicial es proporcionado por \textbf{\texttt{tsp}}.\\

Adem\'as de implementar las funciones que la heur\'istica de Recodio Simulado define, implementa la
heur\'istica de c\'alculo de temperatura incial y el barrido; este \'ultimo para lograr llegar
a un m\'inimo local, dada la soluci\'on final de la heur\'istica.\\

Como la heur\'istica describe lotes de soluciones, se incluye una estructura interna que tiene como
atributos un camino y el promedio del lote.\\

\chapter{Implementaci\'on}
C no es un lenguaje de programaci\'on orientado a objetos, sin embargo, dada su baja abstracci\'on
es f\'acil simularla. Cada una de las clases descritas en el dise\~no est\'an definidas a partir de
estructuras (objetos) que engloban una serie de variables (atributos) y con una serie de funciones
asociadas a ellos (comportamiento).\\

La implementaci\'on de una simulaci\'on de la orientaci\'on a objetos suele basarse en la implementaci\'on
de estructuras que, adem\'as de contar como miembros variables que se considerar\'ian atributos,
tienen como miembro apuntadores a funciones. Sin emabargo, mi implementaci\'on consisti\'o en la definici\'on
de funciones que recib\'ian, en la mayor\'ia de los casos, un apuntador a un objeto de la clase (estructura)
en cuesti\'on.\\

\section{Hilos de ejecuci\'on}
En funci\'on de la maximizaci\'on de ejecuciones del sistema, la idea de la implementaci\'on de la creaci\'on
de hilos de ejecuci\'on no pod\'ia omitirse. El programa fue ejecutado en una computadora de escritorio
con un procesador AMD Ryzen\textsuperscript{TM} 5 1600; este pod\'ia ejecutarse un total de 16 veces
en paralelo y, usando hasta el 70\% de los 16GB de RAM, era posible crear 1,000 hilos de ejecuci\'on.

\subsection{Generador de n\'umeros aleatorios}
Los hilos de ejecuci\'on tra\'ian consigo la limitaci\'on en el uso de variables o funciones globales
que pudieran generar alguna condici\'on de carrera o el establecimiento de una variable como constante
para todos los hilos de ejecuci\'on. Este \'ultimo caso fue un problema presente en mi implementaci\'on;
en un inicio, us\'e las funciones \textbf{\texttt{srand()}} y \textbf{\texttt{rand()}} para generar
los n\'umeros pseudoaleatorios para los algoritmos no deterministas, sin embargo, not\'e que al introducir
los hilos de ejecuci\'on, la primera semilla que se pasaba como par\'ametro a la primera ejecuci\'on
de \textbf{\texttt{srand()}}, era la \'unica usada durante la ejecuci\'on del programa y, por lo tanto,
para todos los hilos creados durante esta.\\

Por ello, introduje una funci\'on nueva: \textbf{\texttt{rand\_r()}}. Esta funci\'on recibe una referencia
a una entero sin signo y genera n\'umeros pseudoaleatorios en el rango [0-RAND\_MAX\footnote{El valor del
  macro RAND\_MAX ser\'a de al menos 32767.}], esto podr\'ia explicar el comportamiento inusual de generaci\'on
de temperaturas iniciales y soluciones casi id\'enticas y con alta regularidad. Es definida como una funci\'on
generadora de n\'umeros pseudoaleatorios d\'ebil, porque el rango del tipo de su semilla no es destacable.

\section{Problemas}
Hace un par de meses realic\'e un proyecto en el lenguaje de programaci\'on C++, esto me introdujo a
conceptos con los que no estaba familiarizado; en general, aquellos relacionados con la asignaci\'on
y liberaci\'on de memoria. Los punteros (o apuntadores) fueron un concepto destacable entre las fuentes de
problemas m\'as comunes durante la depuraci\'on. Sin embargo, C++ provee abstracciones (como la clase
String) que evitaban estos conceptos y se asum\'ian como buenas pr\'acticas o est\'andares del lenguaje.
Es por esto que, a pesar de ser C++ un superconjunto de C, no me familiaric\'e con estos conceptos hasta
poco antes de comenzar el proyecto.

\subsection{Fugas de memoria}
Al no contar con un recolector de basura, es necesario liberar la memoria din\'amica reservada durante
la ejecuci\'on del programa. Producto de la abstracci\'on del programa en clases (estructuras)
la creaci\'on de objetos implicaba una liberaci\'on an\'alogo: aparear cada llamada de \textbf{\texttt{malloc()}}
a \textbf{\texttt{free()}}. Esto se torna una tarea complicada al asignar memoria din\'amica en ciclos. Por ejemplo,
en la implementaci\'on de la clase \textbf{\texttt{database\_loader}}, se asigna una porci\'on de memoria a un
arreglo de apuntadores a objetos  de la clase \textbf{\texttt{city}} (comportamiento provisto por la clase
\textbf{\texttt{city}}) , y para su llenado se itera sobre la funci\'on \textbf{\texttt{callback}} que usa
\textbf{\texttt{sqlite}} para interactuar con los datos que una consulta devuelve; en este ciclo se asigna
memoria para cada ciudad contenida en la base de datos, resulta, por lo tanto, f\'acil perder de vista
cada una de las asignaciones (1092 en total). Se torna una tarea complicada al mezclarla con la ausencia
de la asignaci\'on din\'amica del nombre de las ciudades, que define comportamiento indefinido (guardado en
el stack de la funci\'on), y que sol\'ian desaparecer al intentar imprimirlos en la funci\'on \textbf{\texttt{callback}}:
no me fue claro si el error proven\'ia de la liberaci\'on, en un principio.\\

No mucho tiempo despu\'es encontr\'e una herramienta dise\~nada por Google que pretende ayudar en la localizaci\'on
de las asignaciones que no tuvieron una liberaci\'on apareada (\textit{ASAN}; adem\'as de hacer mucho m\'as
expl\'icitos errores como el acceso a un \'indice no v\'alido de un arreglo o al intentar liberar memoria ya
liberada, entre otros).
Si bien no es completamente certero el decir que el programa no cuenta con fugas de memoria\footnote{\textit{..las
    fugas de memoria pueden ser sustitu\'idas por la determinaci\'on del alto de un programa con las mismas
    consecuencias l\'ogicas.} La determinaci\'on de la ausencia de fugas de memoria implicar\'ia, en ciertos casos,
  la soluci\'on al problema del paro (al lidiar con programas lo suficientemente complejos). ~\cite{samsai}}, la
evaluaci\'on de \textit{ASAN} (Address Sanitizer) y la ejecuci\'on prolongada del programa (con un m\'aximo de 6
horas continuas), sugieren que, en caso de contar con alguna, esta ser\'a despreciable.\\

Durante la implementaci\'on de las pruebas unitarias, un error asociado a la asignaci\'on de memoria surgi\'o,
este era a\'un m\'as obscuro. Se trataba de un error que sirg\'ia cuando el directorio \textbf{\texttt{build}} (usado por
el sistema de construcci\'on \textit{Meson}) del programa no era generado con las banderas de \textit{ASAN};
las pruebas parec\'ian no tener \'exito en casos previamente comprobados. Era a\'un m\'as raro el hecho de que
\textit{ASAN} no detectara alguna fuga o problema asociado con la memoria din\'amica, en general. Este result\'o
ser causado por el uso de \textbf{\texttt{malloc()}} en la inicializaci\'on de variables (arreglos de n\'umeros)
que requer\'ian un estado inicial, antes de ser usados o de determinar qu\'e valores estar\'ian contenidos en
ellas (\textbf{\texttt{calloc()}} tiene este comportamiento). \textit{Valgrind} (otra herramienta para detecci\'on
de fugas de memoria) fue primordial en la detecci\'on de este error.\\

\section{Registros}
Las variables de tipo \textbf{\texttt{register}} son descritas como sigue:
\begin{center}
  \textit{Una declaraci\'on \texttt{\textbf{register}} sugiere al compilador que la variable en cuesti\'on
    ser\'a muy utilizada. La idea es que las variables \texttt{\textbf{register}} ser\'an colocadas en registros
    m\'aquina, lo que resultar\'ia en programas m\'as peque\~nos y r\'apidos. Pero los compiladores son libres
    de ignorar esta sugerencia.~\cite{cpl}}
\end{center}

Consider\'e, as\'i, su introducci\'on en la implementaci\'on del programa (ya que la eficiencia y la optimizaci\'on son
dos conceptos claves). Parec\'ia una tarea simple, pero considerando que la asignaci\'on
de registros \textit{tiene como objetivo el asignar una cantidad finita de registros m\'aquina a un n\'umero
  ilimitado de variables temporales tal que las variables temporales con rangos de vida interferentes sean
  asignados a diferentes registros~\cite{reg}}, suele resolverse con algoritmos (heur\'isticas, en realidad; se
trata de un problema \textit{NP}) que
pretenden encontrar soluciones al problema de coloraci\'on de teor\'ia de gr\'aficas\footnote{El problema de
  coloraci\'on de gr\'aficas puede ser descrito de la siguiente manera: dada una gr\'afica G y un entero positivo
  K, asigna un color a cada v\'ertice de G, usando a lo m\'as K colores, tal que dos v\'ertices adyacentes reciban
  el mismo color.~\cite{reg}}\footnote{\textit{En 1982, Chaitin redujo el problema de coloraci\'on de gr\'aficas, al de
    asignaci\'on de registros , y con ello, probando que la asignaci\'on de registros es un problema NP completo
    tambi\'en}~\cite{reg}~\cite{chaitin}}. El mapeo se hace de v\'ertice a variable temporal, donde las aristas
conectan a las variables cuyos rangos de vida interfieren~\cite{reg}.\\

Que los compiladores eligieran si aceptaban estas sugerencias y que la obtenci\'on de una buena asignaci\'on
fuera reducible a encontrar una soluci\'on para un problema \textit{NP} fueron suficientes para evitar
introducirlos en la estructura del programa.

\chapter{Experimentaci\'on}
El comportamiento del sistema frente a las instancias presentadas no fue el \'optimo, el equilibrio
buscado en los par\'ametros no fue encontrado (esto resultata m\'as notable en la segunda instancia).
Sin embargo, fue posible alcanzar soluciones \'optimas, dada la rapidez del mismo, que involucraba
una ligera p\'erdida en el rendimiento al aumentar ciertos par\'ametros a valores que, normalmente,
ser\'ian considerados inmensos.\\

La calibraci\'on de los par\'ametros no parec\'ia tener una gran repercusi\'on en
las ejecuciones de la primera instancia. Fue hasta la segunda instancia
en la que not\'e que el incremento de algunos de ellos hac\'ian del sistema un programa m\'as
lento, pero con mucho mejores ejecuciones.\\

La experimentaci\'on fue realizada principalmente de manera concurrente en una computadora de escritorio
con especificaciones:
  \begin{itemize} [label=$\scriptstyle{\scriptstyle{\scriptstyle{\scriptstyle{\square}}}}$]
  \item \makebox[5cm][l]{\textit{Procesador:}} AMD Ryzen\textsuperscript{TM} 5 1600 Six-Core @ 3.20GHz (8 Cores / 16 Threads)
    \makebox[3cm][l]{} AMD "Zen" Core Architecture
  \item \makebox[5cm][l]{\textit{Memoria RAM:}} 16GB
  \item \makebox[5cm][l]{\textit{Cach\'es de los procesadores:}} Caché L1 total: 576KB\\
    \makebox[5cm][l]{}    Caché L2 total: 3MB\\
    \makebox[5cm][l]{}    Caché L3 total: 16MB
  \item \makebox[5cm][l]{\textit{Almacenamiento:}} 960GB KINGSTON SA400S3 SSD\\
  \item \makebox[5cm][l]{\textit{OS:}} Arch rolling\\
    \makebox[5cm][l]{\textit{Kernel:}} 6.1.8-zen1-1-zen (x86\_64)
  \end{itemize}


\section{Instancia de 40 ciudades}
Esta instancia no necesit\'o de par\'ametros elevados, ni de una gran cantidad
de ejecuciones para obtener una soluci\'on \'optima. Esta fue obtenida sin la
implementaci\'on de la heur\'istica de temperatura inicial y sin el algoritmo
del barrido. Su obtenci\'on se hizo a trav\'es de 3 mil ejecuciones, dividida
en 3 ejecuciones de 100 hilos cada una, con una duraci\'on aproximada de 3
minutos por ejecuci\'on (el tama\~no del lote tan reducido provocara que las
ejecuciones no tomaran una gran cantidad de tiempo).

\begin{figure}[h!tbp]
  \hspace*{-1.6cm}
  \includesvg[scale=1]{hello}
  \caption{Impresi\'on de las evaluaciones computadas por el lote de una soluci\'on factible.}
\end{figure}
\clearpage
\begin{center}
  \textit{Par\'ametros de la mejor soluci\'on encontrada para la instancia de 40 ciudades.}
\end{center}
\begin{table}[h!]
  \begin{center}
    \begin{tabular}{||l|c||}
      \hline
      \textit{Par\'ametro} & \textit{Valor}\\
      \hline
      Semilla & 102 \\
      Temperatura & 14 \\
      M & 122000 \\
      L & 1200 \\
      \'Epsilon & 0.002 \\
      Phi & 0.95 \\
      \hline
    \end{tabular}
  \end{center}
\end{table}

\begin{figure}[h!tbp]
  \hspace*{-1.6cm}
  \includesvg[scale=1]{best40.svg}
  \caption{Evaluaciones de la mejor soluci\'on encontrada en la instancia de 40 ciudades.}
\end{figure}
\clearpage

\section{Instancia de 150 ciudades}

\begin{center}
  \textit{Par\'ametros de la mejor soluci\'on encontrada para la instancia de 150 ciudades.}
\end{center}
\begin{table}[h!]
  \begin{center}
    \begin{tabular}{||l|c||}
      \hline
      \textit{Par\'ametro} & \textit{Valor}\\
      \hline
      Semilla & 263 \\
      Temperatura & 8 \\
      M & 260000 \\
      L & 12000 \\
      \'Epsilon & 0.000016 \\
      Phi & 0.98 \\
      P & 0.98 \\
      N & 900 \\
      \hline
    \end{tabular}
  \end{center}
\end{table}

\begin{figure}[h!tbp]
  \hspace*{-1.6cm}
  \includesvg[scale=1]{best150.svg}
  \caption{Evaluaciones de la mejor soluci\'on encontrada en la instancia de 150 ciudades.}
\end{figure}

El tam\~no del lote fue un factor muy relevante en la computaci\'on de las soluciones; dejar
64,000 ejecuciones (con un tiempo aproximado de ejecuci\'on de 6 horas) no rindi\'o grandes
frutos, en esta ejecuci\'on se obtuvo como resultado m\'as \'optimo 0.154. Posteriormente,
se ejecut\'o el sistema con una tama\~no de lote de 12000 y se logr\'o superar la soluci\'on
previa, en 3 minutos y ejecutando 200 semillas concurrentes.

\section{Ejecuci\'on de las soluciones \'optimas}

\[\texttt{./build/TSP\_SA -f B\_40.txt -c 40\_instance.txt}\]
\[\texttt{./build/TSP\_SA -f B\_150.txt -c 150\_instance.txt}\]

\chapter{Conclusiones}

Logr\'e observar que mi sistema se comportaba de manera \'optima incluso con par\'ametros
grandes he ah\'i la justificaci\'on de los par\'ametros de la mejor soluci\'on encontrada
para la instancia de 150 ciudades. Por otro lado, un tama\~no de lote reducido (que,
aparentemente, funcionaba mejor con otros sistemas) parec\'ia solo funcionar con la intancia
40 ciudades. La instancia de 150 ciudades tardaba, aproximadamente, 20 veces m\'as en la ejecuci\'on
de una semilla.\\

Dada la experimentaci\'on, considero que el generador de n\'umeros aleatorios fue un factor
clave en la implementaci\'on del sistema. No solo las soluciones parec\'ian falsamente
pseudoaleatorias, sino que tambi\'en la temeperatura inicial era afectada por este problema.
Sin importar la instancia, siempre aparec\'ia una cantidad considerable de manera repetida


\bibliography{ref}{}
\bibliographystyle{plain}
\end{document}
