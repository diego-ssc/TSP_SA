\documentclass[a4paper]{report}

\usepackage{pgf}
\usepackage[utf8]{inputenc}
\usepackage{verbatim}
\usepackage{titling}
\usepackage{booktabs}
\usepackage{enumitem}
\usepackage{qtree}
\usepackage{amssymb}
\usepackage{amsmath}
\usepackage{times}
\usepackage{dsfont}
\usepackage{titling}
\usepackage[spanish]{babel}
% \usepackage{svg}
% \svgpath{../data/plot}


\pretitle{\begin{center}\linespread{1}}
  \posttitle{\end{center}\vspace{0.14cm}}
\preauthor{\begin{center}\Large}
  \postauthor{\end{center}}

\setlength{\droptitle}{-10em}
\title {\textbf {\Large{El problema del agente viajero}}\protect\\
  \large{\textbf{Recocido Simulado}}\protect\\ \vspace{0.4cm}
  \normalsize{\textbf{Seminario de Ciencias de la Computaci\'on B}} \protect\\ \vspace{0.2cm}
  \normalsize{Canek Pel\'aez Vald\'es}}
\date{}
\author{\normalsize Sánchez Correa Diego Sebastián}

\begin{document}
\allowdisplaybreaks
\maketitle
\tableofcontents

\chapter{Introducci\'on}
\section{El problema}

\begin{center}
  \textit{...la pregunta es encontrar, para un conjunto finito de puntos de los
    cuales se conocen las distancias entre cada par, el camino más corto entre
    estos puntos. Por supuesto, el problema es resuelto por un conjunto finito
    de intentos. \textbf{Schrijver (2005)}
    \footnote{Alexander (Lex) Schrijver (4 de Mayo 1948, \'Amsterdam) es 
      un matem\'atico y cient\'ifico en computaci\'on holand\'es,
      profesor de matem\'aticas discretas y optimizaci\'on 
      en la Universidad de \'Amsterdam}}
\end{center}

\section{Heur\'istica}
\chapter{Dise\~no}

El dise\~no sigui\'o una estructura orientada a objetos. Contar con la informaci\'on
de las ciudades en una base de datos relacional, sugiri\'o la implementaci\'on de
un objeto \textbf{\texttt{database\_loader}}. De la msima manera, el contar con soluciones
compuestas por permutaciones de ciudades, hizo claro el uso de las clases
\textbf{\texttt{city}} y \textbf{\texttt{path}} (ciudad y camino, respectivamente).
De la misma manera, se ha planteado el modelado del problema y de la heur\'istica 
como clases compuestas a partir las unidades m\'as fundamentales y con atributos y
comportamiento asociado.\\

\section{La base de datos}
El problema se plante\'o como una base de datos relacional cuyo contenido est\'a
compuesto de las ciudades (1092 en total) y de sus conexiones (que representan
la gr\'afica del problema).\\

Es importante mencionar que la matriz de adyacencias represnetada por la tabla de
conexiones no es cuadrada, sin embargo, el objeto \textbf{\texttt{database\_loader}}
la cargar\'a  a una matriz de enteros (\textbf{\texttt{int}} de C) haci\'endola cuadrada.\\

La tabla de ciudades se carga a un arreglo de apuntadores a objetos de tipo
\textbf{\texttt{city}}. Normalmente se usar\'ia un puntero a un objeto de tipo
\textbf{\texttt{city}}, pero el constructor de la clase, naturalmente, aloja los
objetos en el heap, devolviendo, por lo tanto, un puntero a un objeto;
por ello, el arreglo solo re\'une los punteros de los objetos que
ya se encuentran guardados din\'amicamente; evitando, en conjunto,
una posible ambig\"uedad al tratar con ambos tipos que, en esencia, no
comparten caracter\'isticas adicionales a ser un apuntador a una ciudad.

\section{Las ciudades}
El objeto \textbf{\texttt{city}} tiene como propiedades los campos incluidos en la base de 
datos, con excepci\'on de la poblaci\'on. La clase abstrae el concepto de una
ciudad, omitiendo una representaci\'on total a partir de atributos asociados
(como lo hace la base de datos), sino involucrando en la composici\'on un
comportamiento particular.\\

Esta abstracci\'on se da al decidir que una ciudad proporcione el comportamiento
para calcular la distancia con otra ciudad en conjunto con su uso como unidad
fundamental del problema.\\

Cabe destacar que la clase \textbf{\texttt{city}} tambi\'en es la encargada de crear los
arreglos de ciudades, resultado de una "modularizaci\'on" del dise\~no : ninguna clase
conoce el tama\~no de la estructura \textbf{\texttt{city}}, por lo tanto, ninguna clase (a
excepci\'on de city) es capaz de crear din\'amicamente un arreglo de ciudades; esto
podr\'ia evitarse si se incluyera la declaraci\'on de la estructura dentro del encabezado
(header) de la clase, pero considero se tratar\'ia de una violaci\'on a la encapsulaci\'on
de los datos.

\section{Los caminos}
Al tener la idea de una ciudad ya abstra\'ida en una clase, era natural que los caminos consitieran
de, al menos, un arreglo de ciudades. El primer dise\~no que realic\'e se trataba de una representaci\'on
sin atributos, ni comportamiento; solo un arreglo de ciudades. Poco tiempo despu\'es escrib\'i la
clase \textbf{\texttt{path}}. Esta ahora no solo se trataba de un arreglo de ciudades, sino de
una estructura con atributos y comportamiendo asociados: esto facilit\'o (desde un punto de vista
del disen\~no) la implementaciones de funciones como el intercambio de ciudades dentro de un camino,
el c\'alculo del normalizador y la funci\'on de costo.\\

La creaci\'on de un camino implica el c\'alculo de la distancia m\'axima de esa instancia, as\'i, como el
normalizador y la suma de los costos de sus ciudades. Estos atributos se calculan a partir de operaciones
lineales, por ello, son valores que incializo dentro del constructor de la clase y, si se quiere acceder
a ellos una vez creado al objeto, la operaci\'on se reducir\'a a una consulta lineal del atributo deseado
del objeto. Esta soluci\'on implicaba la disminuci\'on de la cantidad de invocaciones de funciones para
crear una camino v\'alido (dej\'o de ser necesario invocar la funci\'on, seguido de usar un setter del
objeto para guardar el resultado), sin embargo, me imped\'ia utilizar el mismo constructor para hacer copias
de un objeto (usadas en la heur\'istica) si pretend\'ia mantener el tiempo de ejecuci\'on en un estado
\'optimo, por supuesto.\\

La implementaci\'on de la funci\'on copia de la clase \textbf{\texttt{path}}, entonces, no invoca al
constructor de la clase, sino que accede a las propiedades del objeto par\'ametro y aquellas que eran
computadas linealmente, se asociar\'ian como copias al nuevo objeto.

\section{TSP}
El problema del agente viajero fue, como lo han sido todas las estructuras del proyecto, modelado siguiendo
un dise\~no orientado a objetos, sin embargo, no ser\'ia un error afirmar que este no era su prop\'osito inicial.
La clase comenz\'o siendo un objeto central de ejecuci\'on que administrar\'ia la invocaci\'on y liberaci\'on
de todas las clases necesarias, incluso, ser\'ia la encargada de ejecutar la heur\'istica.\\

La clase desempe\~na el papel de marcar el inicio de la ejecuci\'on del problema, creando un objeto
\textbf{\texttt{database\_loader}}, proveyendo la informaci\'on computada en este.
Tiene como atributo un camino (el \'unico creado en la ejecuci\'on del programa; con el constructor
de la clase, al menos) que tambi\'en ser\'a usado para la heur\'istica. Por ello, una ejeuci\'on,
no estar\'a completa sin una invocaci\'on a la clase \textbf{\texttt{sa}}.

\section{Recocido Simulado}


\chapter{Implementaci\'on}
\section{Hilos de ejecuci\'on}
\subsection{Generador de n\'umeros aleatorios}
\chapter{Experimentaci\'on}

\begin{table}[h!]
  \begin{center}
    \begin{tabular}{||l|c||}
      \hline
      \textit{Par\'ametro} & \textit{Valor}\\
      \hline
      Semilla & 102 \\
      Temperatura & 14 \\
      M & 122000 \\
      L & 1200 \\
      \'Epsilon & 0.002 \\
      Phi & 0.95 \\
      \hline
    \end{tabular}
  \end{center}
\end{table}
% \begin{figure}[h!tbp]
%   \hspace*{-1.6cm}
%   \includesvg[scale=1]{results_1}
% \end{figure}


% \begin{figure}[h!tbp]
%   \hspace*{-1.6cm}
%   \includesvg[scale=1]{results_2}
% \end{figure}

\chapter{Conclusiones}

\section{Bibliograf\'ia}


\end{document}
